\documentclass[11pt, a4paper]{scrreprt}

% TODO: export in Layout-File
% TODO: parse into new layout!
% TODO: create table of bibliography
% TODO: create table of contents  --> the real one!

%% ------- LAYOUT ------- %%
\usepackage[T1]{fontenc}
\usepackage[utf8]{inputenc}
\usepackage[ngerman]{babel}
\usepackage{amsmath}
\usepackage{amssymb}
\usepackage{selinput}
\SelectInputMappings{
 adieresis={ä},
 germandbls={ß},
}
% Wörter mit Umlauten werden getrennt
%% ------------ END LAYOUT -------%

\title{Implementierung einer Smartphone-Anwendung zum Austausch verschlüsselter Daten mit einer Cloud}
\begin{document}
\titlepage
\maketitle
\tableofcontents

%% ------------------------------------------------------ %%
%% ------------------------ BEGIN ---------------------- %%
%% -------------------------------------------------------%% 


%% -------------------- EINLEITUNG ------------------ %%
\part{Einleitung}
	% TODO: Text über Sicherheit, Privatsphäre, Bedürnisse, Internet, Überwachung, Smartphones ..
\chapter{Motivation}
Am Deutschen Elektronen Synchrotron, im folgenden DESY, werden bisher wichtige und sensible Dokumente über ein Programm Namens Dropbox gesichert und verwaltet. Dropbox bietet eine Plattformunabhänginge Möglichkeit Dokuente Online abzuspeichern und von jedem anderen Standort über ein Internetfähiges Gerät die Daten wieder zu öffnen.
Da das DESY über eine eigene Cloud-Infrastruktur verfügt, sollen in Zukunft alle wichtigen Dokumente voll verschlüsselt auf diesem Server abgelegt werden.

\chapter{Zielsetzung}
	% TODO: Dieser Block ist eventuell mit in Motivation enthalten
\chapter{Verwandte Arbeiten}
\chapter{Diese Arbeit}
\section{Inhaltlicher Aufbau}
\section{Veränderung ggü. Anforderung}

%% ---------------------- GRUNDLAGEN ----------------- %%
\part{Grundlagen}
\chapter{Android}
\section{Zusammenhang Kryptographie}

\chapter{Kryptographie}
\section{Grundlagen}
	% TODO: Ziele, Begriffserklärung: kryptographie, kryptoanalsyse, kryptologie, symetrisch, asymetrisch, 
\section{Verschlüsselung}
\subsection{Symetrische Verfahren}
\subsection{Asymetrische Verfahren}

\section{Hash-Funktionen}
\section{Digitale Signature}
	% TODO: Begriffserklärung: integrität, authentifizierung, authentizierung usw. ..
\subsection{Public Key Infrastruktur}
\section{Schlüsselvereinbarung}
\subsection{Diffie Hellmann}
\subsection{ElGamal}
\section{Schlüsselgenerierung}

%% ------------------------ VALIDIERUNG --------------------- %%
\part{Validierung}
	% TODO: Erläuterung des Testgerätes
\chapter{Verschlüsselungsverfahren}
\chapter{Hashfunktionen}

%% ------------------------ IMPLEMENTIERUNG ---------------- %%
\part{Implementierung}
\chapter{Entwurf}

%% ----------------------------- TEST -------------------------- %%
\part{Test}
\chapter{Validierung}
\chapter{Testverfahren}

%% -------------------------- ZUSAMMENFASSUNG -------------- %%
\part{Zusammenfassung und Ausblick}
\chapter{Zusammenfassung}
\chapter{Ausblick}

%Glossar
%Literatur
%Diagramme
%Tabellen
%eigenständigkeitsverklärung

\end{document}