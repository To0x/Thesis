\documentclass[11pt, a4paper]{scrreprt}

% TODO: export in Layout-File
% TODO: parse into new layout!
% TODO: create table of bibliography
% TODO: create table of contents  --> the real one!

%% ------- LAYOUT ------- %%
\usepackage[T1]{fontenc}
\usepackage[utf8]{inputenc}
\usepackage[ngerman]{babel}
\usepackage{amsmath}
\usepackage{amssymb}
\usepackage{selinput}
\SelectInputMappings{
 adieresis={ä},
 germandbls={ß},
}
% Wörter mit Umlauten werden getrennt


%% ------------ END LAYOUT -------%
\title{Implementierung einer Smartphone-Anwendung zum Austausch verschlüsselter Daten mit einer Cloud}
\begin{document}
\titlepage
\maketitle
\tableofcontents

%% -------------------- EINLEITUNG ------------------ %%
\part{Einleitung}
\chapter{Motivation}
\chapter{Zielsetzung}
TODO: Dieser Block ist eventuell mit in Motivation enthalten
\chapter{Verwandte Arbeiten}
\chapter{Diese Arbeit}
\section{Inhaltlicher Aufbau}
\section{Veränderung ggü. Anforderung}

%% ---------------------- GRUNDLAGEN ----------------- %%
\part{Grundlagen}
\chapter{Android}
\section{Zusammenhang Kryptographie}

\chapter{Kryptographie}
\section{Grundlagen}
% TODO: Ziele, Begriffserklärung: kryptographie, kryptoanalsyse, kryptologie, symetrisch, asymetrisch, 
\section{Verschlüsselung}
\subsection{Symetrische Verfahren}
\subsection{Asymetrische Verfahren}

\section{Hash-Funktionen}
\section{Digitale Signature}
% TODO: Begriffserklärung: integrität, authentifizierung, authentizierung usw. ..
\subsection{Public Key Infrastruktur}
\section{Schlüsselvereinbarung}
\subsection{Diffie Hellmann}
\subsection{ElGamal}
\section{Schlüsselgenerierung}


\part{Validierung}
% TODO: Erläuterung des Testgerätes
\section{Verschlüsselungsverfahren}
\section{Hashfunktionen}

\part{Implementierung}
\section{Entwurf}

\part{Test}
\section{Validierung}
\section{Testverfahren}

\part{Zusammenfassung und Ausblick}
\section{Zusammenfassung}
\section{Ausblick}

%Glossar
%Literatur
%Diagramme
%Tabellen
%eigenständigkeitsverklärung

\end{document}